\documentclass[10pt, a4paper]{proc}
\usepackage[utf8]{inputenc}
\usepackage{multicol}
\usepackage{setspace}
\usepackage{ragged2e}
\usepackage{fancyhdr}
\usepackage{titlesec}
\usepackage[left=2.5cm,right=2.5cm,
    top=2.5cm,bottom=2.5cm]{geometry}

\justifying
\fancyhf{} % clear all header and footers
\renewcommand{\headrulewidth}{0pt} % remove the header rule
\cfoot{\vskip -1.5cm \thepage}
\pagestyle{fancy}
\linespread{0.84}
\setlength{\columnsep}{0.5cm}
\setcounter{page}{73}
\renewcommand{\thesection}{\Roman{section}}
\titleformat{\section}{\large\centering\sc}{\thesection. }{0cm}{}[]

\title{
 \begin{spacing}{1.2}
  \textbf{\LARGE{The standardization of intelligent computer\\systems as a key challenge of the current stage\\of development of artificial intelligence\\technologies}}
 \end{spacing}
}

\author{
 Vladimir Golenkov\\\textit{Belarusian State University}\\\textit{of Informatics and Radioelectronics}\\Minsk, Belarus\\golen@bsuir.by\\
 \and
 Vladimir Golovko\\\textit{Brest State Technical University}\\Brest, Belarus\\vladimir.golovko@gmail.com
 \and
 Natalia Guliakina\\\textit{Belarusian State University}\\\textit{of Informatics and Radioelectronics}\\Minsk, Belarus\\guliakina@bsuir.by
 \and
 Viktor Krasnoproshin\\\textit{Belarusian State University}\\Minsk, Belarus\\krasnoproshin@bsu.by
}

\begin{document}
 \maketitle
 \textbf{
 \vskip 4cm
  \quad\textit{Abstract}—This work is devoted to the consideration of the most important factor providing semantic compatibility of intelligent computer systems and their components – standardization of intelligent computer systems, as well as standardization of methods and tools of their design.\\
  \quad\textit{Keywords}—intelligent computer system; integrated technology for the development of intelligent computer systems; knowledge base of an intelligent computer system; educational activities in the field of artificial intelligence; research activities in the field of artificial intelligence; development of artificial intelligence technologies; engineering activities in the field of artificial intelligence; convergence of scientific disciplines in the field of artificial intelligence; convergence of models for the representation and processing of knowledge in intelligent computer systems; general theory of intelligent computer systems; convergence of various methods and means of developing intelligent computer systems; semantic compatibility of intelligent computer systems and their components; hybrid intelligent computer system; semantic representation of knowledge; standard of intelligent computer systems.
 }
 \justifying
 \section{Introduction}
 \normalsize
 \quad The report at the OSTIS-2019 conference [1] examined the key problem for the current stage of development of artificial intelligence technologies: to ensure \textit{\textbf{semantic compatibility}} of intelligent computer systems and \textit{semantic compatibility} of various components of such systems (various types of knowledge that are part of knowledge bases, various problem solving models, various\\~\\\vskip 3.94cm components of a multimodal interface).\\
 \indent This work is devoted to the consideration of the most important factor providing \textit{semantic compatibility} of intelligent computer systems and their components – \textbf{\textit{standardization of intelligent computer systems}}, as well as standardization of methods and tools of their design.\\
 \indent The basis of the proposed approach to ensuring a high level of \textit{learnability} and \textit{semantic compatibility} of intelligent computer systems, as well as to the development of \textit{\textbf{standard of intelligent computer systems}} is the unification of \textit{\textbf{sense representation of knowledge}} in the memory of such systems and the construction of \textit{\textbf{global sense space of knowledge}}.
 \section{The current state of artificial intelligence as a field of scientific and technical activities}
 \normalsize
 \textbf{\textit{artifical Intelligence}}
 \vspace{-0.3cm}
 \setlength{\columnsep}{-6cm}
 \begin{multicols}{2}
  :=

  \columnbreak
  \noindent[An interdisciplinary (transdisciplinary) field of scientific and technical activity aimed at the development and application of \textbf{\textit{intelligent computer systems}}, which provide automation of various spheres of human activity]
 \end{multicols}
 \begin{multicols}{2}
  :=

  \columnbreak
  \noindent[The field of human activity aimed at (1) building the theory of \textit{intelligent computer systems} (2) development of technologies (methods and tools) for their design, (3) development of applied \textit{intelligent computer systems}]
 \end{multicols}
 \(\Leftarrow\) decomposition*:\\
 \{
 \begin{itemize}
  \item educational activities in the field of artificial intelligence
  \item research activities in the field of artificial intelligence
  \item development of integrated technologies in the field of artificial intelligence
  \item development of a set of intelligent computer systems
  \item business in the field of artificial intelligence
 \end{itemize}
 \}\\
 \(\Rightarrow\) feature*:\\
 \indent\([\)
 A feature of artificial intelligence as a field of canscientific and technical activity is that it has a pronounced interdisciplinary, interdisciplinary, integration, collective character. Success here is largely determined by the consistency of actions and the level of compatibility of the results of these actions]\\

 \indent Despite the presence of \textbf{\textit{serious scientific results}} in the field of \textit{artificial intelligence}, the pace of development of \textbf{\textit{market of intelligent computer systems}} is not so impressive. There are several reasons for this:
 \begin{itemize}
  \item there is a big gap between scientific research in the field of \textit{artificial intelligence} and the creation of industrial complex technologies for the development of \textit{intelligent computer systems}. Scientific research in the field of \textit{artificial intelligence} is mainly concentrated on the development of new methods for intelligent problems solving;
  \item these studies are fragmented and the need for their integration and the creation of a general formal theory of intelligent systems is not recognized, that is, there is a "Babylonian crowd" of various models, methods and tools used in \textit{artificial intelligence} in the absence of awareness of the problem of providing their compatibility. Without a solution to this problem, neither a general theory of intelligent systems can be created, nor, therefore, an integrated technology for \textit{intelligent computer systems} development, available to both engineers and \textbf{\textit{experts}};
  \item the specified integration of models and methods of artificial intelligence is very complicated, because it is interdisciplinary in nature;
  \item intelligent systems as design objects have a significantly higher level of complexity compared to all the technical systems that humanity has dealt with;
  \item as a consequence of the above, there is a big gap between scientific research and engineering practice in this area. This gap can only be filled by creating an evolving technology for \textit{intelligent computer systems} development, the evolution of which is carried out through the active cooperation of scientists and engineers;
  \item the quality of the development of applied intelligent systems to a large extent depends on the mutual understanding of experts and knowledge engineers. Knowledge engineers, not knowing the intricacies of the application field, can make serious mistakes in the developed knowledge base. The mediation of knowledge engineers between experts and the developed knowledge base significantly reduces the quality of the developed intelligent computer systems. To solve this problem, it is necessary that the language of knowledge representation in \textit{knowledge base} be "convenient" not only for intelligent systems and knowledge engineers, \textit{\textbf{but also for experts}}.
 \end{itemize}
 The current state of artificial intelligence technologies can be characterized as follows [2]–[7]:
 \begin{itemize}
  \item There is a large set of particular \textit{artificial intelligence technologies} with appropriate tools, but there is no general \textit{theory of intelligent systems} and, as a result, there is no general \textit{complex design technology for intelligent computer systems};
 \end{itemize}
 The development of \textit{artificial intelligence technologies} is substantly hampered by the following sociomethodological circumstances:
 \begin{itemize}
  \item The high social interest in the results of work in the field of \textit{artificial intelligence} and the great complexity of this science give rise to superficiality and dishonesty in the development and advertising of various applications. Serious science is mixed up with irresponsible marketing, conceptual and terminological sloppiness and illiteracy, throwing in new absolutely unnecessary effective terms, confusing the essence of the matter, but creating the illusion of fundamental novelty.
  \item The interdisciplinary nature of research in the field of \textit{artificial intelligence} substantially complicates this research, since work at the intersections of scientific disciplines requires high culture and qualifications.
 \end{itemize}
 To solve the above problems of the development of \textit{artifical intelligence technologies}:
 \begin{itemize}
  \item  Continuing to develop new formal models for \textit{intelligent problems} solving and to improve existing models (logical, neural networks, production), it is necessary to ensure \textbf{compatibility} of these models both among themselves and with traditional models for solving problems that are not among the intellectual tasks. In other words, we are talking about the development of principles for organizing \textit{hybrid intelligent computer systems} that provide solutions to \textbf{complex tasks} that require sharing and in unpredictable combinations of a wide variety of types of knowledge and a wide variety of problem solving models.
  \item A transition is needed from the eclectic construction of complex intelligent computer systems using different types of knowledge and various types of problem solving models to their deep integration, when the same presentation models and knowledge processing models are implemented in different systems and subsystems in the same way.
  \item It is necessary to reduce the distance between the modern level of \textit{theory of intelligent systems}    and the practice of their development.
 \end{itemize}
 \noindent\textbf{\textit{artificical intelligence}}\\
 \(\Rightarrow\) \textit{development trends}*:\\
 \{
 \begin{itemize}
  \item \([\)Erasing interdisciplinary barriers between different areas of research in \textit{artificial intelligence}.]
  \item \([\)Transferring the emphasis from scientific research aimed at studying the phenomenon of intelligence and building formal models of intelligent processes to creating of industrial complex technology for \textit{intelligent computer systems design}.]
 \end{itemize}
 \}\\
 \(\Rightarrow\) \textit{development problems}*:\\
 \{
 \begin{itemize}
  \item \([\)Lack of motivation among scientists to integrate their research results in the field of \textit{artificial intelligence} within the framework of the general theory of intelligent systems.]
  \item \([\)Insufficient level of motivation and consistency for the transition from the theory of intelligent systems to \textbf{\textit{integrated technology for intelligent computer systems design}}, ensuring their semantic compatibility.]
  \item \([\)Lack of effective interaction between various activities that provide development of \textit{artificial intelligence} (educational activities, research activities, technology development, engineering, the business in the field of artificial intelligence).]
 \end{itemize}
 \}\\
 \(\Rightarrow\) \textit{consequence}*:\\
 \([\)The consequence of these problems is that the current state of artificial intelligence technologies does not provide the required development of the market for artificial intelligent systems.]\\
 \(\Rightarrow\) \textit{what to do}*:\\
 \([\)For the development of artificial intelligence technologies, close interaction between practical engineers, developers of new technologies and scientists in the field of artificial intelligence is necessary.]\\~\\
 \noindent\textbf{\textit{artificial intelligence}}\\
 \(\Rightarrow\) \textit{key development tasks}*:\\
 \{
 \begin{itemize}
  \item \([\)the convergence of various areas of scientific research in the field of artificial intelligence in order to create \textit{general theory of intelligent systems}]
  \item \([\)integration of the existing variety of models, methods and tools for developing various components of intelligent systems into \textit{unified integrated technology}, providing intelligent systems development for automating various fields of activity]
  \item \([\)ensuring of \textit{semantic compatibility} of the developed intelligent computer systems]
  \item \([\)integration and coordination of various activities that ensure the sustainable development of artificial intelligence:
  \begin{enumerate}
   \item[--] educational activities aimed at training of specialists capable of effectively participating in the development of artificial intelligence;
   \item[--] artificial intelligence research activity;
   \item[--] activities aimed at the development of artificial intelligence technologies;
   \item[--] applied intelligent systems engineering;
   \item[--] of a business aimed at organizing and financially supporting all the above types of activities and, first of all, at introducing the developed systems.
  \end{enumerate}
  \quad]\\
  \}
 \end{itemize}
 \quad The problem of creating a fast-growing market for semantically compatible intelligent systems is a challenge addressed to specialists in the field of artificial intelligence, requiring overcoming the "Babylonian crowd" in all its manifestations, the formation of a high culture of negotiability and a unified, consistent form of representation of collectively accumulated, improved and used knowledge.\\
 \indent Scientists working in the field of artificial intelligence should ensure the convergence of the results of different areas of artificial intelligence and build a general theory of intelligent computer systems and \textit{integrated technology for semantically compatible intelligent computer systems design}, including the appropriate \textit{\textbf{standards of intelligent computer systems}} and their components.\\
 \indent Engineers developing intelligent computer systems should collaborate with scientists and participate in the development of integrated technology for intelligent computer systems design.\\
 \section{The convergence of different activities in the field of artificial intelligence}
  \normalsize
  \quad The further development of artificial intelligence affects all forms and directions of activity in this area. We list the main areas of convergence in the field of artificial intelligence [8]–[10]:
 \begin{itemize}
  \item The convergence of various disciplines in the training of specialists in the field of artificial intelligence in order to form a holistic picture of \textbf{problems} in the field of \textit{artificial intelligence};
  \item The convergence of various scientific studies in the field of \textit{artificial intelligence} in order to build \textbf{\textit{general theory of intelligent computer systems}};
  \item Convergence of development methods and tools for various \textit{intelligent computer systems} to create \textit{\textbf{integrated technology for intelligent computer systems}} development, available to a wide range of engineers;
 \end{itemize}
\end{document}
