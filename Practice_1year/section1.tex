%Пример

\begin{SCn}
\begin{small}

% 24 билет
\scnheader{решатель задач ostis-системы}

\scnidtf{совокупность всех навыков, которыми обладает ostis-система на текущий момент времени}
\scnrelto{семейство подмножеств}{навык}
\scntext{примечание}{Предлагаемый в рамках Технологии OSTIS подход к построению решателей задач позволяет обеспечить их
модифицируемость, что, в свою очередь, позволяет ostis-системе при необходимости легко приобретать
новые навыки, модифицировать (совершенствовать) уже имеющиеся, и даже избавляться от некоторых
навыков с целью повышения производительности системы. Таким образом, имеет смысл говорить не о
жестко фиксированном решателе задач, который разрабатывается один раз при создании первой версии
системы и далее не меняется, а о совокупности навыков, фиксированной в каждый текущий момент времени,
но постоянно эволюционирующей.}
\scnsuperset{объединенный решатель задач ostis-системы}
\scnaddlevel{1} 
\scnidtf{полный решатель задач ostis-системы}
\scnidtf{интегрированный решатель задач ostis-системы}

\scnidtf{решатель задач ostis-системы, реализующий все ее функциональные возможности, как основные, так
и вспомогательные}
\scntext{пояснение}{В общем случае \textit{объединенный решатель задач ostis-системы}, решает задачи, связанные с:
    \begin{scnitemize}
        \item обеспечением основных функциональных возможностей системы (например, решение явно сформулированных задач по требованию пользователя);
        \item обеспечением корректности и оптимизацией работы самой ostis-системы (перманентно на протяжении всего жизненного цикла ostis-системы);
        \item обеспечением повышения квалификации конечных пользователей и разработчиков ostis-системы;
        \item обеспечением автоматизации развития и управления развитием ostis-системы.
    \end{scnitemize}}
    \scnrelboth{тождественно}
{интегрированный решатель задач}
\scnaddlevel{1}
\scnsuperset{sc-модель интегрированного решателя задач}
\scnaddlevel{1}
\scnidtf{\textit{неатомарный абстрактный sc-агент}, являющийся результатом объединения всех \textit{абстрактных sc-агентов}, входящих в состав конкретной ostis-системы, в один.}
\scnaddlevel{-2}

\scnaddlevel{-1}
\scnsuperset{гибридный решатель задач ostis-системы}
\scnaddlevel{1} 
\scnidtf{решатель задач ostis-системы, реализующий две и более модели решения задач}






\scnheader{решатель задач ostis-системы}
\scnsuperset{решатель задач вспомогательной ostis-системы}
\scnaddlevel{1}
\scnsuperset{решатель задач интерфейса компьютерной системы}
\scnaddlevel{1}
\scnsubdividing{решатель задач пользовательского интерфейса компьютерной системы;решатель задач интерфейса компьютерной системы с другими компьютерными
системами;решатель задач интерфейса компьютерной системы с окружающей средой}
\scnaddlevel{-1}
\scnsuperset{решатель задач ostis-подсистемы поддержки проектирования компонентов определенного класса}
\scnaddlevel{1}
\scnsuperset{решатель задач ostis-подсистемы поддержки проектирования баз знаний}
\scnaddlevel{1}
\scnsuperset{решатель задач повышения качества базы знаний}
\scnaddlevel{1}
\scnsuperset{решатель задач верификации базы знаний}
\scnaddlevel{1}
\scnsuperset{решатель задач поиска и устранения некорректностей в базе знаний}
\scnsuperset{решатель задач поиска и устранения неполноты}
\scnaddlevel{-1}
\scnsuperset{решатель задач оптимизации структуры базы знаний}
\scnsuperset{решатель задач выявления и устранения информационного мусора}
\scnaddlevel{-2}
\scnsuperset{решатель задач ostis-подсистемы поддержки проектирования решателей задач ostis-систем}
\scnaddlevel{1}
\scnsubdividing{решатель задач ostis-подсистемы поддержки проектирования программ обработки знаний;решатель задач ostis-подсистемы поддержки проектирования агентов
обработки знаний}
\scnaddlevel{-2}
\scnsuperset{решатель задач подсистемы управления проектирования компьютерных систем и их компонентов}
\scnaddlevel{-1}
\scnsuperset{решатель задач самостоятельной ostis-системы}
\scnheader{решатель задач ostis-системы}
\scnsuperset{решатель задач с использованием хранимых методов}
\scnaddlevel{1}
\scnidtf{решатель, способный решать задачи тех классов, для которых на данный момент времени известен
соответствующий метод решения}
\scnsuperset{решатель задач на основе нейросетевых моделей
}
\scnsuperset{решатель задач на основе генетических алгоритмов}
\scnsuperset{решатель задач на основе императивных программ}
\scnaddlevel{1}
\scnsuperset{решатель задач на основе процедурных программ
}
\scnsuperset{решатель задач на основе объектно-ориентированных программ
}
\scnaddlevel{-1}
\scnsuperset{решатель задач на основе декларативных программ
}
\scnaddlevel{1}
\scnsuperset{решатель задач на основе логических программ
}
\scnsuperset{решатель задач на основе функциональных программ
}
\scnaddlevel{-2}
\scnsuperset{решатель задач в условиях, когда метод решения задач данного класса в текущий момент времени не
известен
}

\scnidtf{решатель, реализующий стратегии решения задач, позволяющие породить метод решения задачи,
который в текущий момент времени не известен ostis-системе}
\scnidtf{решатель, использующий для решения задач метаметоды, соответствующие более общим классам
задач по отношению к заданной}
\scnidtf{решатель задач, позволяющий породить метод, который является частным по отношению какому-либо
известному ostis-системе методу и интерпретируется соответствующей машиной обработки знаний}
\scnsuperset{решатель, реализующий стратегию поиска путей решения задачи в глубину}
\scnsuperset{решатель, реализующий стратегию поиска путей решения задачи в ширину}
\scnsuperset{решатель, реализующий стратегию проб и ошибок
}
\scnsuperset{решатель, реализующий стратегию разбиения задачи на подзадачи}
\scnsuperset{решатель, реализующий стратегию решения задач по аналогии
}
\scnsuperset{решатель, реализующий концепцию интеллектуального пакета программ}
\scnheader{решатель задач ostis-системы}



\scnsuperset{решатель задач информационного поиска}
\scnaddlevel{1}
\scnsubdividing{решатель задач поиска информации, удовлетворяющей заданным критериям;решатель задач поиска информации, не удовлетворяющей заданным критериям}
\scnaddlevel{-1}
\scnsuperset{решатель явно сформулированных задач}
\scnaddlevel{1}
\scnidtf{решатель задач, для которых явно сформулирована цель}
\scnsuperset{решатель задач поиска или вычисления значений заданного множества величин
}
\scnsuperset{решатель задач установления истинности заданного логического высказывания в рамках заданной
формальной теории
}
\scnsuperset{решатель задач формирования доказательства заданного высказывания в рамках заданной
формальной теории}
\scnsuperset{машина верификации ответа на указанную задачу}
\scnsuperset{машина верификации решения указанной задачи}
\scnaddlevel{1}
\scnsuperset{машина верификации доказательства заданного высказывания в рамках заданной
формальной теории}
\scnaddlevel{-2}
\scnsuperset{решатель задач классификации сущностей}
\scnaddlevel{1}
\scnsuperset{машина соотнесения сущности с одним из заданного множества классов}
\scnsuperset{машина разделения множества сущностей на классы по заданному множеству признаков}
\scnaddlevel{-1}
\scnsuperset{решатель задач синтеза информационных конструкций
}
\scnaddlevel{1}
\scnsuperset{решатель задач синтеза естественно-языковых текстов}
\scnsuperset{решатель задач синтеза изображений}
\scnsuperset{решатель задач синтеза сигналов
}
\scnaddlevel{1}
\scnsuperset{решатель задач синтеза речи}
\scnaddlevel{-2}
\scnsuperset{решатель задач анализа информационных конструкций}
\scnaddlevel{1}
\scnsuperset{решатель задач анализа естественно-языковых текстов}
\scnaddlevel{1}
\scnsuperset{решатель задач понимания естественно-языковых текстов}

\scnsuperset{решатель задач верификации естественно-языковых текстов}
\scnaddlevel{-1}
\scnsuperset{решатель задач анализа изображений}
\scnaddlevel{1}
\scnsuperset{решатель задач сегментации изображений}
\scnsuperset{решатель задач понимания изображений}
\scnaddlevel{-1}
\scnsuperset{решатель задач анализа сигналов
}
\scnaddlevel{1}
\scnsuperset{решатель задач анализа речи}
\scnaddlevel{1}

\scnsuperset{решатель задач понимания речи}


\scnaddlevel{-3}


\scnheader{решатель задач ostis-системы}
%\scnaddlevel{1}
\scnsuperset{информационно-поисковая машина}
\scnaddlevel{1}
\scnrelboth{тождественно}
{машина обработки знаний}
\scnaddlevel{-1}
\scnsuperset{служебные операции обработки знаний}
\scnaddlevel{1}
    \scntext{пояснение}{Например, операция сборки мусора, выявления противоречий в базе знаний}
\scnrelfromvector{принципы проектирования}{по возможности быть предметно независимыми; по возможности меньше ориентироваться на
фиксированную форму представления фрагментов базы знаний, на работу с
которыми она ориентирована; одна операция может состоять из ряда подпрограмм на выбранном языке
программирования;каждая операция должна самостоятельно проверять полноту соответствия условия
инициирования конструкции, имеющейся в памяти системы на данный момент;если в процессе работы операция генерирует в памяти какие-либо временные
конструкции, то при завершении работы она обязана удалить всю информацию, использование которой в системе более нецелесообразно; при необходимости операции можно объединять в группы для решения
более сложных задач;инициатором запуска какой-либо операции может быть как непосредственно
пользователь системы, так и другая операция}






\scnheader{машина обработки знаний}
\scnsuperset{sc-агент}
\scnaddlevel{1}
\scnsuperset{абстрактный sc-агент}
\scnaddlevel{1}
\scnsuperset{неатомарный абстрактный sc-агент}
\scnaddlevel{1}
\scnidtf{абстрактный
sc-агент, который декомпозируется на коллектив более простых абстрактных
sc-агентов, каждый из которых в свою очередь может быть как атомарным
абстрактным sc-агентом, так и неатомарным абстрактным sc-агентом.}
\scnaddlevel{-1}
\scnsuperset{атомарный абстрактный sc-агент}
\scnaddlevel{-1} 

\scnaddlevel{1}
\scnidtf{некоторый класс функционально эквивалентных sc-агентов, разные экзмепляры (т.е. представители) которого могут быть реализованы по-разному}
\scnrelfromvector{спецификация}{указание ключевых sc-эдементов это sc-агента, т.е. тех sc-элементов, хранимых в sc-памяти, которые дя данного sc-агента являются "точками опоры";формальное описание условий инициирования данного sc-агента, т.е.тех ситуаций в sc-памяти, которые инициируют деятеельность данного sc-объекта;формальное описание первичного условия инициирования данного sc-объекта, т.е. такой ситуации  в sc-памяти, которая побуждает sc-агента перейти в активное состояние и начать проверку наличия своего полного условия инициирования; строгое, полное однозначно пониманиемое описание деятельности данного sc-агента, оформленное пр  помощи каких-либо понятных, общепринятых средств, не требующих специального изучения, например, на естественном языке; описание, результатов выполнения данного sc-агента}
\scnaddlevel{-1} 


\scnidtf{открытая система, помещенная в неекоторую среду, обладающая собственным поведением, удовлетворяющим некоторым экстремальным принципам}
\scnrelfromvector{способность}{воспринимать информацию из внешней среды с ограниченным разрешением; обрабатывать ее на основе собственных ресурсов; взаимодействовать с другими агентами; действовать на среду в течение некоторого времени, преследую собственные цели}


\scnaddlevel{-1}
\scntext{пояснение}{Под машиной обработки знаний будем понимать совокупность интерпретаторов всех навыков, составляющих некоторый решатель задач. С учетом многоагентного подхода к обработке информации, используемого
в рамках Технологии OSTIS, машина обработки знаний представляет собой sc-агент (чаще всего – неатомарный sc-агент), в состав которого входят более простые sc-агенты, обеспечивающие интерпретацию соответствующего множества методов. Таким образом, машина обработки знаний в общем случае представляет собой иерархическую систему sc-агентов.}
\scnheader{решатель задач ostis-системы}
\scnrelto{описывают особенности}{общие классы задач}
\scnaddlevel{1}
\scnsubdividing{задачи на доказательство; задачи на преобразование; задачи на описание; задачи на исследование; задачи классификации}
\scntext{пояснение}{Каждая задача в общем случае может и не относиться к какому-либо классу задач полностью, однако содержать элементы сразу нескольких классов. Соответственно, при решении таких задач сочетаются стратегии и модели, присущие нескольким классам задач.}
\scnheader{решатель задач информационной системы}
%\scnaddlevel{1}
    \scnrelfromvector{специфические требования}{решатель задач должен быть легко модифицируемым, то есть трудоемкость внесения изменений в уже разработанный решатель задач должна быть минимальной.;для того чтобы информационная система имела возможность
анализировать и оптимизировать имеющийся
решатель задач, интегрировать в его состав новые компоненты (в том числе самостоятельно), оценивать
важность тех или иных компонентов и применимость их для решения той или иной задачи,
спецификация решатля задач должна быть описана языком, понятным системе.; рефлексивность}
\scnsuperset{возможности информационной системы}
\scnaddlevel{1}
\scnhaselement{рефлексивность}
\scnidtf{возможность информационной системы анализировать (верифицировать, корректировать,
оптимизировать) собственные компонент.}
\scnaddlevel{-1}
\scnheader{решатель задач информационной системы нового поколения}
\scnsuperset{принципы построения решателя задач нового поколения}
\scnaddlevel{1}
\scnrelfromvector{разбиение}{в качестве основы для построения РЗ
предлагается использовать многоагентный
подход;процесс решения любой задачи предлагается декомпозировать на логически атомарные действия; РЗ предлагается рассматривать как
иерархическую систему, состоящую из нескольких взаимосвязанных уровней;для обеспечения рефлексивности проектируемых ИС предлагается записывать всю информацию о РЗ и решаемых им задачах при помощи SC-кода в той же базе знаний, что и собственно предметные знания системы;при проектировании РЗ как иерархической системы на каждом из уровней предлагается использовать компонентный подход;предлагается строить средства автоматизации и информационной поддержки разработчиков РЗ с использованием технологии
OSTIS;коммуникацию агентов предлагается
осуществлять по принципу доски объявлений, однако в отличие от классического
подхода в роли сообщений выступают спецификации в общей семантической памяти выполняемых агентами действий;в роли внешней среды для агентов выступает та же семантическая память, в которой
формулируются задачи и посредством которой
осуществляется взаимодействие агентов;спецификация каждого агента описывается средствами SC-кода в той же семантической памяти;синхронизацию деятельности агентов
предполагается осуществлять на уровне выполняемых ими процессов, направленных на
решение тех или иных задач в семантической
памяти;каждый информационный процесс в любой момент времени имеет ассоциативный доступ к необходимым фрагментам базы знаний,
хранящейся в семантической памяти, за исключением фрагментов, заблокированных другими
процессами,исключая необходимость хранения
каждым агентом информации о внешней среде}
\scnaddlevel{1}
    \scnrelto{пояснение}{РЗ -- решатель задач.}
\scnaddlevel{-1}
\scnsuperset{основные уровни детализации решателя задач}
\scnaddlevel{1}
\scnrelfromvector{разбиение}{уровень самого решателя задач;уровень неатомарных sc-агентов, входящих в состав решателя задач, в том числе более частных решателей задач;уровень атомарных sc-агентов; уровень scp-программ или программ, реализованных на уровне платформы
интерпретации sc-моделей.}
\scnaddlevel{1}
    \scntext{пояснение}{Такая иерархия уровней обеспечивает возможность:
    \begin{scnitemize} 
\item поэтапного проектирования решателя задач с постепенным повышением степени
детализации от верхнего уровня к нижнему
\item проектирования, отладки и верификации компонентов на разных уровнях независимо от других уровней, что существенно
упрощает задачу построения и модификации
решателя задач за счет снижения накладных расходов
\itemобеспечить ее модифицируемость и возможность согласованного
использования различных моделей решения задач в рамках одного решателя задач.
\end{scnitemize}
}
\scnaddlevel{-1}
\scnaddlevel{-2}


\scnheader{план действий интеллектуальной системы}
\scnexplanation{результат решения задачи, сценарий, частично-упорядоченная совокупность действий}
\scnsuperset{Методы поиска плана действий интеллектуальной системы}
\scnaddlevel{1}
\scnhaselement{Метод планирования общего решателя задач}
\scnaddlevel{1}
\scntext{пояснение}{Первая наиболее известная модель планировщика.}
\scnrelfromvector{основные принципы поиска}{анализ целей и средств;рекурсивное решение задач}
\scnrelfromvector{стандартные задачи}{преобразовать объект А в объект В; уменьшить различие D между А и В; применить оператор f к объекту А}
\scnaddlevel{-1}
\scnaddlevel{-1}
\scnsuperset{Задача планирования действий интеллектуальной системы}
\scnaddlevel{1}
\scnhaselement{задача автоматического синтеза программ}
\scnaddlevel{1}
\scntext{пояснение}{Поиск плана дейсвтий в этом сучае трактуется как поиск прграммы, решающей ту задачу, формулировка которой поступает на вход системы планирования, что приводит к схеме формулировка задачи -> теорема -> доказательство -> программа.}
\scnrelfromvector{стандартные задачи}{преобразовать объект А в объект В; уменьшить различие D между А и В; применить оператор f к объекту А}
\scnsubdividing{трансляция описания исходной задачи в формулу логического исчисления, которая для этого исчисления выступает как формулировка теоремы, подлежащей доказательству; поиск вывода в используемом исчислении; извлечение из найденного вывода траектории, которая преобразуется в нужную программу}
\scnrelfromvector{проблема извлечения программы из доказательства}{специфика проблемной области, в которой решается задача; характер используемого логического исчисления и применяемых в нем методов доказательства теоремы.}
\scnaddlevel{-1}

% 30 билет

\scnheader{Библиотека многократно используемых компонентов OSTIS}\\
\scnidtf{Библиотека OSTIS}
\scntext{примечание}{Библиотека многократно используемых компонентов включает в себя как сами используемые компоненты, так и средства их спецификации, а также средства поиска компонентов по различным критерям, и, таким образом, представляют собой целую подсистему со своей базой знаний и решателем задач.}
\scnrelfromset{декомпозиция}{
    Библиотека платформ интерпретации sc-моделей компьютерных систем;
    Библиотека многократно используемых компонентов баз знаний ostis-систем;
    Библиотека многократно используемых внутренних агентов ostis-систем\\
    \scnaddlevel{1}
    \scnrelfromset{декомпозиция}{
        Библиотека sc-агентов информационного поиска;
        Библиотека sc-агентов погружения интегрируемого знания в базу знаний;
        Библиотека sc-агентов выравнивания онтологии интегрируемого знания с основной онтологией текущего состояния базы знаний;
        Библиотека sc-агентов планирования решения явно сформулированных задач;
        Библиотека sc-агентов логического вывода;
        Библиотека sc-агентов обнаружения и удаления информационного мусора;
        Библиотека координирующих sc-агентов;
        Библиотека sc-моделей языков программирования выского уровня и соответствуюших им интерпретаторов
    }
    \scnaddlevel{-1};
    Библиотека многократно используемых методов, интерпретируемых ostis-системами\\
    \scnaddlevel{1}
    \scnrelfrom{подсистема}{
        Библиотека многократно используемых программ обработки sc-текстов\\
        \scnrelfrom{подсистема}{Библиотека многократно используемых scp-программ}
    }
    \scnaddlevel{-1};
    Библиотека многократно используемых компонентов интерфейсов ostis-систем\\
    \scnaddlevel{1}
        \scnrelfromset{декомпозиция}{
            Библиотека описаний внешних языков;
            Библиотека редакторов внешних информационных конструкций;
            Библиотека трансляторов в sc-память;
            Библиотека трансляторов из sc-памяти во внешнее представление;
            Библиотека визуализаторов;
            Библиотека сред общения;
            Библиотека элементов управления пользовательским интерфейсом
        }
    \scnaddlevel{-1};
    Библиотека многократно используемых встраиваемых ostis-систем
}

\end{small}
\end{SCn}
