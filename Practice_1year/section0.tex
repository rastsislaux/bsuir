%Пример

\begin{SCn}
\begin{small}

\scnheader{Часть 2 Учебной дисциплины "Представление и обработка информации в интеллектуальных системах"{}}
\scnrelfromlist{библиографическая ссылка}{Стандарт OSTIS;Материалы конференций OSTIS;Журнал "Онтология проектирования"{};Справочник по Искусственному интеллекту в трех томах;Энциклопедический словарь по информатике для начинающих;Толковый словарь по Искусственному интеллекту;Статья "Электронные обучающие системы с использованием интеллектуальных технологий"\\
\scnaddlevel{1}
    \scntext{URL}{http://raai.org/library/tolk/aivoc.html}
\scnaddlevel{-1}}

\scnrelfromvector{аттестационные вопросы}{
    %Вопрос 24 по Части 2 Учебной дисциплины "Представление и обработка информации в интеллектуальных системах"{};
    Вопрос 29 по Части 2 Учебной дисциплины "Представление и обработка информации в интеллектуальных системах"{};
    Вопрос 30 по Части 2 Учебной дисциплины "Представление и обработка информации в интеллектуальных системах"{}
}

%\scnheader{Вопрос 24 по Части 2 Учебной дисциплины "Представление и обработка информации в интеллектуальных системах"{}}
%\scnidtf{Уточнение приницпов построения решателя задач в интеллектуальной системе нового поколения, основанной на смысловом представлении знаний. Действия, задачи и методы, выполняемые таким решателем задач.}
%\scnrelfromlist{библиографическая ссылка}{Стандарт OSTIS}

\scnheader{Вопрос 29 по Части 2 Учебной дисциплины "Представление и обработка информации в интеллектуальных системах"{}}
\scnidtf{Интерфейс интеллектуальной компьютерной системы, основанный на смысловом представлении знаний как интеллектуальный решатель интерфейсных задач. Понятие интерфейсной задачи. Рецепторно-эффекторный уровень взаимодействия интеллектуальной компьютерной системы с внешней средой. Мультирецепторная (мультисенсорная) интеграция и рецепторно-эффекторная координация. Языковое взаимодействие интеллектуальной компьютерной системы с внешними субъектами (обмен сообщениями). Интерфейсная мультимодальность, проблема понимания. Пользовательский интерфейс.}
\scnrelfromlist{библиографическая ссылка}{
    Стандарт OSTIS;
    Microsoft.WindowsDA-2011el\\
    \scnaddlevel{1}
        \scnidtf{Microsoft,. Windows desktop applications - guidelines. — 2011. https://docs.microsoft.com/ru-RU/windows/apps/design/controls/.}
    \scnaddlevel{-1};
    Maybury.M.IntellUserInterfIntro-1998art\\
    \scnaddlevel{1}
        \scnidtf{Maybury, Mark. Intelligent user interfaces: An introduction / Mark Maybury, W. Wahlster // Readings in Intelligent User Interfaces. — 1998. — P. 1–13.}
    \scnaddlevel{-1};
    Ehlert.P.IntellUserInterfIAS-2003art\\
    \scnaddlevel{1}
        \scnidtf{Ehlert, Patrick. Intelligent user interfaces: Introduction and survey. — 2003.}
    \scnaddlevel{-1}
}

\scnheader{Вопрос 30 по Части 2 Учебной дисциплины "Представление и обработка информации в интеллектуальных системах"{}}
\scnidtf{Библиотека многократно используемых компонентов интеллектуальных компьютерных систем, основанных на смысловом представлении знаний.}
\scnrelfromlist{библиографическая ссылка}{
    Библиотека многократно используемых компонентов OSTIS\\
    \scnaddlevel{1}
        \scniselement{раздел стандарта OSTIS}
    \scnaddlevel{-1};
    Pfister.C.ComponentSoftware-1997art\\
    \scnaddlevel{1}
        \scnidtf{C., Phister. Component Software / Phister C. — 1997}
    \scnaddlevel{-1};
    Szyperski.C.CompSoftBOOP-2002book\\
    \scnaddlevel{1}
        \scnidtf{Szyperski, C. Component Software: Beyond Object-oriented Programming / C. Szyperski, D. Gruntz, S. Murer. ACM Press Series. — ACM Press, 2002.}
    \scnaddlevel{-1};
    Khan.A.PerspecStudyISFCBD-2015art\\
    \scnaddlevel{1}
        \scnidtf{Khan, Asif. A perspective study of intelligent system for component based development / Asif Khan, Md Mottahir Alam, Mohammad Shariq // International Journal of Computer Applications. — 2015. — Vol. 117. — P. 11–17.}
    \scnaddlevel{-1};
    USGenServAdm.2.101.Definitions-reg\\
    \scnaddlevel{1}
        \scnidtf{Government, U. S. Federal acquisition regulation.}
    \scnaddlevel{-1}
}

\end{small}
\end{SCn}