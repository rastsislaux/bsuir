%Пример

\begin{SCn}
\begin{small}

\scnheader{Часть 2 Учебной дисциплины "Представление и обработка информации в интеллектуальных системах"{}}
\scnrelfromlist{библиографическая ссылка}{Стандарт OSTIS;Материалы конференций OSTIS;Журнал "Онтология проектирования"{};Справочник по Искусственному интеллекту в трех томах;Энциклопедический словарь по информатике для начинающих;Толковый словарь по Искусственному интеллекту;Статья "Электронные обучающие системы с использованием интеллектуальных технологий"\\
\scnaddlevel{1}
    \scntext{URL}{http://raai.org/library/tolk/aivoc.html}
\scnaddlevel{-1}}

\scnrelfromvector{аттестационные вопросы}{
    Вопрос 24 по Части 2 Учебной дисциплины "Представление и обработка информации в интеллектуальных системах"{};
    Вопрос 29 по Части 2 Учебной дисциплины "Представление и обработка информации в интеллектуальных системах"{};
    Вопрос 30 по Части 2 Учебной дисциплины "Представление и обработка информации в интеллектуальных системах"{}
}

\scnheader{Вопрос 24 по Части 2 Учебной дисциплины "Представление и обработка информации в интеллектуальных системах"{}}
\scnidtf{Уточнение приницпов построения решателя задач в интеллектуальной системе нового поколения, основанной на смысловом представлении знаний. Действия, задачи и методы, выполняемые таким решателем задач.}
\scnrelfromlist{библиографическая ссылка}{
    Стандарт OSTIS;
    Голенков.В.В.АгентОриентМоделМСРСРЗИС-2020ст\\
    \scnaddlevel{1}
        \scnidtf{Голенков, В. В. Агентно-ориентированные модели, методика и средства разработки совместимых решателей задач интеллектуальных систем//Программные продукты и системы / В. В. Голенков, Д. В. Шункевич. — 2020. — P. 404–412.}
    \scnaddlevel{-1};
    Голенков.В.В.МоделРешенЗадачИС-2015ст\\
    \scnaddlevel{1}
        \scnidtf{Модели решения задач в интеллектуальных системах / В. В. Голенков [et al.]. — Минск: БГУИР, 2015. — 70 p.}
    \scnaddlevel{-1};
    Поспелов.Д.А.ИсскусИнтелМиМ-1990кн\\
    \scnaddlevel{1}
        \scnidtf{Поспелов, Д. А. Исскуственный интеллект. Книга 2. Модели и методы / Д. А. Поспелов. — Радио и связь, 1990. — 304 p.}
    \scnaddlevel{-1};
    Якимчик.С.В.ПринПострРешатЗадачИС-2014ст\\
    \scnaddlevel{1}
        \scnidtf{Якимчик, С. В. Принципы построения решателей задач в прикладных интеллектуальных системах / С. В. Якимчик, Д. В. Шункевич // Информационные технологии и системы 2014 (ИТС 2014) : материалы международной научной конференции, БГУИР, Минск, Беларусь, 29 октября 2014 г. – Information Technologies and Systems 2014 (ITS 2014) : Proceeding of The International Conference, BSUIR, Minsk, 29th October 2014 / редкол. : Л. Ю. Шилин [и др.]. – Минск : БГУИР, 2014. – С. 160–161.}
    \scnaddlevel{-1}
}

%\scnheader{Вопрос 29 по Части 2 Учебной дисциплины "Представление и обработка информации в интеллектуальных системах"{}}
%\scnidtf{Интерфейс интеллект компьютерной системы, основанный на смысловом представлении знаний как интеллектуальный решатель интерфейсных задач. Понятие интерфейсной задачи. Рецепторно-эффекторный уровень взаимодействия интеллектуальной компьютерной системы с внешней средой. Мультирецепторная (мультисенсорная) интеграция и рецепторно-эффекторная координация. Языковое взаимодействие интеллектуальной компьютерной системы с внешними субъектами (обмен сообщениями). Интерфейсная мультимодальность, проблема понимания. Пользовательский интерфейс.}
%\scnrelfromlist{библиографическая ссылка}{Стандарт OSTIS}

\scnheader{Вопрос 30 по Части 2 Учебной дисциплины "Представление и обработка информации в интеллектуальных системах"{}}
\scnidtf{Библиотека многократно используемых компонентов интеллектуальных компьютерных систем, основанных на смысловом представлении знаний.}
\scnrelfromlist{библиографическая ссылка}{
    Библиотека многократно используемых компонентов OSTIS\\
    \scnaddlevel{1}
        \scniselement{раздел стандарта OSTIS}
    \scnaddlevel{-1};
    Pfister.C.ComponentSoftware-1997ст\\
    \scnaddlevel{1}
        \scnidtf{C., Phister. Component Software / Phister C. — 1997. — P. 3, 44.}
    \scnaddlevel{-1};
    Manning.C.IntrodToInforRetrieval-2008кн\\
    \scnaddlevel{1}
        \scnidtf{Manning C., Raghavan P., Schütze H. Introduction to Information Retrieval. -- Cambridge University Press, 2008}
    \scnaddlevel{-1};
    Euzenat.J.OntolMatch-2013кн\\
    \scnaddlevel{1}
        \scnidtf{Euzenat, Jerome. Ontology matching / Jerome Euzenat, Pavel Shvaiko. — 2nd ed. — Heidelberg (DE): Springer-Verlag, 2013}
    \scnaddlevel{-1};
    Шункевич.Д.В.МногоагПодходКПострМашинОбрЗнОсновСемСет-2013ст\\
    \scnaddlevel{1}
        \scnidtf{Шункевич Д.В. — Многоагентный подход к построению машин обработки знаний на основе семантических сетей // Кибернетика и программирование. – 2013. – № 1. – С. 37 - 45.}
    \scnaddlevel{-1};
    Jones.R.GarbagCollecHandbookAAMM-2016кн\\
    \scnaddlevel{1}
        \scnidtf{Jones, R. The Garbage Collection Handbook: The Art of Automatic Memory Management / R. Jones, A. Hosking, E. Moss. "International Perspectives on Science, Culture and Society". — CRC Press, 2016.}
    \scnaddlevel{-1};
    Голенков.В.В.СеманТехнПроектИнтелСистСеманАссоцКомп-2019ст\\
    \scnaddlevel{1}
        \scnidtf{Голенков В.В. -- Семантические технологии проектирования интеллектуальных систем и семантические ассоциативные компьютеры / В. В. Голенков, Н. А. Гулякина, И. Т. Давыденко, Д. В. Шункевич // Доклады БГУИР. - 2019. - №3.}
    \scnaddlevel{-1};
    Голенков.В.В.СеманТехнКомпонПроектСистУпрЗн-2015ст\\
    \scnaddlevel{1}
        \scnidtf{Голенков В.В. -- Семантическая технология компонентного проектирования систем, управляемых знаниями / В. В. Голенков, Н. А. Гулякина - Минск, 2015}
    \scnaddlevel{-1};
    Давыденко.И.Т.МоделМетодСредствРазрГибридБЗОССМИК-2018дис\\
    \scnaddlevel{1}
        \scnidtf{Давыденко И.Т. -- Модели, методика и средства разработки гибридных баз знаний на основе семантической совместимости многократно используемых компонентов / И. Т. Давыденко - Минск, 2019.}
    \scnaddlevel{-1};
    Голенков.В.В.ПроектОткрСемнТехнКомпПроектИСЧ2УМП-2014ст\\
    \scnaddlevel{1}
        \scnidtf{Голенков В.В. -- Проект открытой семантической технологии компонентного проектирования интеллектуальных систем. Часть 2: Унифицированные модели проектирования / В. В. Голенков, Н. А. Гулякина // Научный журнал "Онтология проектирования". - 2014.}
    \scnaddlevel{-1};
    Ивашенко.В.П.МоделАлгорИнтегрЗнОснОднорСемСет-2015дис\\
    \scnaddlevel{1}
        \scnidtf{Ивашенко В.П. -- Модели и алгоритмы интеграции знаний на основе однородных семантических сетей / В. П. Ивашенко. - Минск, 2015}
    \scnaddlevel{-1};
    Голенков.В.В.ОнтолПроектГибрСемСовмИСОснСмыслПредЗн-2019ст\\
    \scnaddlevel{1}
        \scnidtf{Голенков В.В. -- Онтологическое проектирования гибридных семантически совместимых интеллектуальных систем на основе смыслового представления знаний / В. В. Голенков, Н. А. Гулякина, И. Т. Давыденко, Д. В. Шункевич, А. П. Еремеев // Научный журнал "Онтология проектирования". -- 2019.}
    \scnaddlevel{-1};
    Шункевич.Д.В.МашинОбрЗнИнтелМетаСистПоддержПроектИнтелСист-2014ст\\
    \scnaddlevel{1}
        \scnidtf{Шункевич Д.В. -- Машина обработки знаний интеллектуальной метасистемы поддержки проектирования интеллектуальных систем / Д. В. Шункевич // Материалы IV международной научно-технической конференции OSTIS. -- Минск, 2014.}
    \scnaddlevel{-1};
    Wooldridge.M.IntroMAS-2002кн\\
    \scnaddlevel{1}
        \scnidtf{Wooldridge, M. An Introduction to MultiAgent Systems / M. Wooldridge. — Wiley, 2002.}
    \scnaddlevel{-1}
}

\end{small}
\end{SCn}
