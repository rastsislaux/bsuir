\begin{SCn}
\begin{small}

% 24 билет

\scnheader{Голенков.В.В.МоделРешенЗадачИС-2015ст}
\scnrelfrom{тип}{пособие}
\scnrelfromlist{ключевой знак}{решатель задач;sc-агент;база знаний;семантическая сеть}
\scntext{цитата}{Интеллектуальный решатель задач является важнейшей частью любой интеллектуальной системы, т. к.
именно возможностями решателя задач определяется функционал системы в
целом, возможность давать ответы на нетривиальные вопросы пользователя и
способность решать различные задачи.}
\scnaddlevel{1}
    \scnrelto{пояснение}{решатель задач}
\scnaddlevel{-1}
\scntext{цитата}{В
связи с этим становится важным формирование общих принципов организации
решателей задач, позволяющих осуществить интегрирование различных
моделей решения задач, как существующих, так и новых. Это предоставляет
возможность использовать в конкретной интеллектуальной системе любые
модели решения задач.}
\scnaddlevel{1}
    \scnrelto{пояснение}{решатель задач}
\scnaddlevel{-1}
\scntext{цитата}{Семантическая сеть – это ориентированный граф, вершины которого –
понятия, а дуги – отношения между ними}
\scnaddlevel{1}
    \scnrelto{пояснение}{семантическая сеть}
\scnaddlevel{-1}
\scntext{цитата}{База знаний – совокупность знаний предметной области, записанная на
машинный носитель в форме, понятной эксперту и пользователю (обычно на
некотором языке, приближенном к естественному). Параллельно такому
«человеческому» представлению существует база знаний во внутреннем
«машинном» представлении}
\scnaddlevel{1}
    \scnrelto{пояснение}{база знаний}
\scnaddlevel{-1}


\scnheader{Якимчик.С.В.ПринПострРешатЗадачИС-2014ст}
\scnrelfrom{тип}{статья}
\scnrelfromlist{ключевой знак}{решатель задач;sc-агент}
\scntext{цитата}{В связи с необходимостью каждой прикладной интеллектуальной системы решать не только задачи информационного поиска, но и задачи
логического вывода и генерации новых знаний,
в рамказ проекта OSTIS разрабатывается технология компонентного проектирования интеллектуальных решателей задач.}
\scntext{цитата}{Модель решения задачи основана на многоагентном подходе, что позволяет использовать одних и тех же интеллектуальных агентов для решения разных задач и классов задач.}
\scntext{цитата}{В общем случае решатель задач представляет собой графодинамическую sc-машину, в состав которой входит ассоциативная перестраиваемая (графодинамческая) память (sc-память) и множество агентов.}


\scnheader{Поспелов.Д.А.ИсскусИнтелМиМ-1990кн}
\scnrelfrom{тип источника}{справочник}
\scnrelfromlist{ключевой знак}{план действий;общий решатель задач}
\scntext{цитата}{Поиск плана действий возникает в ИС лишь тогда, когла она сталкивается с нестандартной ситуацией, для которой нет заранее известного набора дейсвтий, приводящих к нужной цели.}
\scnaddlevel{1}
    \scnrelto{пояснение}{план действий}
\scnaddlevel{-1}
\scntext{цитата}{В целом стратегия общего решателя задач осуществляет  обратный поиск -- от заданной цели В к требуемому средству ее достижения С, используя редукцию исходной задачи <А, В> к задачам <А, С> и <С, В>.}
\scnaddlevel{1}
    \scnrelto{пояснение}{метод планирования общего решателя задач}
\scnaddlevel{-1}
\scntext{цитата}{В ОРЗ молчаливо предполагается независимость различий друг от друга, откуда следует гарантия, что уменьшение одних различий не приведет к увеличению других.}
\scnaddlevel{1}
    \scnrelto{пояснение}{общий решатель задач}
\scnaddlevel{-1}


\scnheader{Голенков.В.В.АгентОриентМоделМСРСРЗИС-2020ст}
\scnrelfrom{тип}{статья}
\scnrelfromlist{ключевой знак}{решатель задач;многоагентная система;sc-агент;семантическая сеть}

\scntext{цитата}{Главной проблемой, рассматриваемой в работе, является низкая согласованность принципов, лежащих в основе реализации различных моделей решения задач. Как следствие, существенно затруднено
одновременное использование различных моделей решения задач в единой системе при выполнении
одной и той же комплексной задачи, практически невозможно повторно использовать технические решения, реализованные в какой-либо системе, кроме того, фактически нет комплексных методик и инструментальных средств, способных обеспечить процесс разработки решателей задач на всех этапах}
\scnaddlevel{1}
    \scnrelto{пояснение}{решатель задач}
\scnaddlevel{-1}
\scntext{цитата}{Решатель задач предлагается рассматривать как
иерархическую систему, состоящую из нескольких взаимосвязанных уровней. Такой
подход позволяет обеспечивать возможность проектирования, отладки и верификации компонентов на разных уровнях независимо от других уровней.
}
\scnaddlevel{1}
    \scnrelto{пояснение}{принципы постоения решателя задач}
\scnaddlevel{-1}
\scntext{цитата}{В качестве основы для построения РЗ
предлагается использовать многоагентный
подход, который позволит обеспечить возможность построения параллельных асинхронных
систем, имеющих распределенную архитектуру, повысить модифицируемость и производительность разработанных РЗ.
}
\scnaddlevel{1}
    \scnrelto{пояснение}{принципы постоения решателя задач}
\scnaddlevel{-1}

\scntext{цитата}{Процесс решения любой задачи предлагается декомпозировать на логически атомарные действия, что также позволит обеспечить
совместимость и модифицируемость РЗ.
}
\scnaddlevel{1}
    \scnrelto{пояснение}{принципы постоения решателя задач}
\scnaddlevel{-1}
\scntext{цитата}{При проектировании РЗ как иерархической системы на каждом из уровней предлагается использовать компонентный подход, что
позволит существенно снизить сроки разработки и повысить надежность РЗ за счет использования отлаженных компонентов. Для реализации такого подхода предлагается разработать в рамках метасистемы IMS[(11)] библиотеку компонентов РЗ различного
уровня, а также методику построения и модификации РЗ, учитывающую наличие такой библиотеки.
}
\scnaddlevel{1}
    \scnrelto{пояснение}{принципы постоения решателя задач}
\scnaddlevel{-1}
\scntext{цитата}{Предлагается строить средства автоматизации и информационной поддержки разработчиков РЗ с использованием технологии
OSTIS, то есть и с использованием моделей,
методики и средств, предлагаемых в данной работе. Такой подход обеспечит высокие темпы
развития указанных средств, а также существенно повысит эффективность средств информационной поддержки, позволяя строить
указанные средства как часть интеллектуальной Метасистемы IMS.
}
\scnaddlevel{1}
    \scnrelto{пояснение}{принципы постоения решателя задач}
\scnaddlevel{-1}
\scntext{цитата}{Коммуникацию агентов предлагается
осуществлять по принципу доски объявлений, однако в отличие от классического
подхода в роли сообщений выступают спецификации в общей семантической памяти выполняемых агентами действий. Такой подход
позволяет исключить необходимость разработки специализированного языка для обмена
сообщениями, обеспечить модифицируемость
всей системы за счет обезличенности общения,
а также возможность формулировать задачи в декларативном ключе, то есть не указывать
явно способ решения задачи.
}
\scnaddlevel{1}
    \scnrelto{пояснение}{принципы постоения решателя задач}
\scnaddlevel{-1}
\scntext{цитата}{В роли внешней среды для агентов выступает та же семантическая память, в которой
формулируются задачи и посредством которой
осуществляется взаимодействие агентов. Такой подход обеспечивает унификацию среды
для различных систем агентов, что, в свою очередь, обеспечивает их совместимость.
}
\scnaddlevel{1}
    \scnrelto{пояснение}{принципы постоения решателя задач}
\scnaddlevel{-1}
\scntext{цитата}{Спецификация каждого агента описывается средствами SC-кода в той же семантической памяти, что позволяет минимизировать
число специализированных средств, необходимых для спецификации как языковых, так и инструментальных агентов.
}
\scnaddlevel{1}
    \scnrelto{пояснение}{принципы постоения решателя задач}
\scnaddlevel{-1}
\scntext{цитата}{Под
sc-моделью интегрированного РЗ понимается
коллектив всех sc-агентов, входящих в состав
заданной ostis-системы, воспринимаемый как
единое целое.
}
\scnaddlevel{1}
    \scnrelto{пояснение}{sc-модель интегрированного решателя задач}
\scnaddlevel{-1}


% https://oberoncore.ru/_media/library/c_pfister_component_software.pdf
\scnheader{Pfister.C.ComponentSoftware-1997ст}
\scnrelfrom{тип}{статья}
\scnrelfromlist{ключевой знак}{комопнент;компоненто-ориентированное программирование}

% The advent of component software may be the most important new development in the software industry
% since the introduction of high-level programming languages. Component software combines the advantages
% of custom software and standard software. It enables solutions which are better evolvable, i.e., which scale
% better, are more readily maintainable, can be extended over time, and can be modernized incrementally.
\scntext{аннотация}{Появление компонентного программного обеспечения может быть самым важным событием в индустрии программного обеспечения с момента появления языков программирования высокого уровня. Компонентное программное обеспечение сочетает в себе преимущества заказного и стандартного программного обеспечения. Это позволяет создавать решения, более приспособолены к эволюции, т. е. которые лучше масштабируются, которые легче обслуживать, которые можно расширять с течением времени и постепенно модернизировать.}

% A component is a unit of composition with a contractually specified interface and explicit context
% dependencies only. Components can be deployed independently of each other and are subject to
% composition by third parties.
\scntext{цитата}{Компонент — это единица композиции с указанным в контракте интерфейсом и явным контекстом зависимостей. Компоненты могут быть развернуты независимо друг от друга и подлежат композиции третьими лицами.}
\scnaddlevel{1}
    \scnrelto{определение}{Компонент}
    \scnrelto{пояснение}{Библиотека OSTIS}
\scnaddlevel{-1}

\scnheader{Manning.C.IntrodToInforRetrieval-2008кн}
\scnrelfrom{тип}{книга}
\scnrelfromlist{ключевой знак}{информационный поиск}

% Information retrieval (IR) is finding material (usually documents) of
% an unstructured nature (usually text) that satisfies an information need
% from within large collections (usually stored on computers).

\scntext{цитата}{Информационный поиск – это поиск материала (как правило, документов), имеющих неструктурированный характер (обычно текст), который удовлетворяет потребность в информации из больших коллекций (обычно хранящихся на компьютерах).}
\scnaddlevel{1}
    \scnrelto{определение}{Информационный поиск}
    \scnrelto{пояснение}{Библиотека sc-агентов информационного поиска}
\scnaddlevel{-1}


% http://book.ontologymatching.org/glossary.html
\scnheader{Euzenat.J.OntolMatch-2013кн}
\scnrelfrom{тип}{книга}
\scnrelfromlist{ключевой знак}{онтологии;выравнивание онтологий}

% Ontology Matching aims at being a reference book that presents currently available work in the topic in a uniform framework. In particular, though we use the word ontology, the work and the techniques considered in this book can equally be applied to database schema matching, catalogue integration, XML schema matching and other related problems. The objectives of the book include presenting (i) the state of the art and (ii) the latest research results in ontology matching by providing a detailed account of matching techniques and matching systems in a systematic way from theoretical, practical and application perspectives. The main emphasis of this book is thus on technical solutions for matching.

\scntext{аннотация}{Ontology Matching стремится быть справочником, в котором представлены доступные в настоящее время работы по теме в единой структуре. В частности, хотя мы используем слово «онтология», работа и методы, рассматриваемые в этой книге, в равной степени могут быть применены к сопоставлению схемы базы данных, интеграции каталогов, сопоставлению XML-схем и другим связанным проблемам. В задачи книги входит представление (i) современного уровня техники и (ii) последних результатов исследований в области сопоставления онтологий путем систематического подробного описания методов сопоставления и систем сопоставления с теоретической, практической и прикладной точек зрения. Таким образом, основное внимание в этой книге уделяется техническим решениям для сопоставления.}

% Matching is the process of finding relationships or correspondences between entities of different ontologies.

\scntext{цитата}{Сопоставление онтологий - процесс нахождения отношений или соответствий между сущностями различных онтологий.}
\scnaddlevel{1}
    \scnrelto{определение}{Сопоставление онтологий}
    \scnreltolist{пояснение}{Библиотека sc-агентов погружения интегрируемого знания в базу знаний;Библиотека sc-агентов выравнивания онтологии инетгрируемого знания с основной онтологией текущего состояния базы знаний}
\scnaddlevel{-1}

% Alignment is a set of correspondences between two or more (in case of multiple matching) ontologies (by analogy with molecular sequence alignment). The alignment is the output of the matching process.
\scntext{цитата}{Выравнивание онтологий - множество соответствий между двумя или более (в случае множественного сопоставления) онтологиями. Выравнивание - результат процесса сопоставления.}
\scnaddlevel{1}
    \scnrelto{определение}{Выравнивание онтологий}
    \scnreltolist{пояснение}{Библиотека sc-агентов погружения интегрируемого знания в базу знаний;Библиотека sc-агентов выравнивания онтологии инетгрируемого знания с основной онтологией текущего состояния базы знаний}
\scnaddlevel{-1}

% Ontology merging is the creation of a new ontology from two, possibly overlapping, source ontologies. The initial ontologies remain unaltered. The merged ontology is assumed to contain the knowledge of the initial ontologies, e.g., consequences of each ontology are consequences of the merge. This concept is closely related to that of schema integration in databases.

\scntext{цитата}{Объединение онтологий - создание новой онтологии из двух, возможно пересекающихся, исходных онтологий. Исходные онтологии остаются незименными. Объединённная онтология должна содержать знания исходных отнологий... Эта концепция тесно связана с интеграцией схем в базах данных.}
\scnaddlevel{1}
    \scnrelto{определение}{Объединение онтологий}
    \scnreltolist{пояснение}{Библиотека sc-агентов погружения интегрируемого знания в базу знаний;Библиотека sc-агентов выравнивания онтологии инетгрируемого знания с основной онтологией текущего состояния базы знаний}
    \scnreltoset{сравнение}{Объединений онтологий;Интеграция схем в базах данных}
\scnaddlevel{-1}

% Ontology integration is the inclusion in one ontology of another ontology and assertions expressing the glue between these ontologies, usually as bridge axioms. The integrated ontology is assumed to contain the knowledge of both initial ontologies. Contrary to merging, the first ontology is unaltered while the second one is modified.

\scntext{цитата}{Интеграция онтологий - это включение в одну онтологию другой онтологии и утверждениях, выступающих в качестве "клея" между двумя онтологиями, обычно это набор аксиом-мостов. Предполагается, что интегрированная онтология содержит знания двух исходных онтологий. В отличии от объединения, первая онтологии остаётся неизменной, в то время как вторая изменяется.}
\scnaddlevel{1}
    \scnrelto{определение}{Интеграция онтологий}
    \scnreltolist{пояснение}{Библиотека sc-агентов погружения интегрируемого знания в базу знаний;Библиотека sc-агентов выравнивания онтологии инетгрируемого знания с основной онтологией текущего состояния базы знаний}
    \scnreltoset{сравнение}{Объединение онтологий;Интеграция онтологий}
\scnaddlevel{-1}

% Bridge axioms or articulation axioms are formulas, in an ontology language, that express the alignments such that it is possible to integrate the entities of an ontology within one another. Bridge axioms are the basis for ontology merging when the ontologies are expressed in the same language.
\scntext{цитата}{Аксиомы-мосты или аксиомы артикуляции - это формулы на языке онтологии, которые выражают выравнивания таким образом, что становится возможно интегрировать сущности онтологии между собой. Аксиомы-мосты являются основой для объединения онтологий, когда онтологии выражены на одном языке.}
\scnaddlevel{1}
    \scnrelto{определение}{Аксиомы-мосты}
    \scnreltolist{пояснение}{Библиотека sc-агентов погружения интегрируемого знания в базу знаний;Библиотека sc-агентов выравнивания онтологии инетгрируемого знания с основной онтологией текущего состояния базы знаний}
\scnaddlevel{-1}

\scnheader{Wooldridge.M.IntroMAS-2002кн}
\scnrelfrom{тип}{книга}
\scnrelfromlist{ключевой знак}{агент;мультиагентная система}
%This is the first textbook to be explicitly designed for use as a course text for an undergraduate/graduate course on multi-agent systems. Assuming only a basic understanding of computer science, this text provides an introduction to all the main issues in the theory and practice of intelligent agents and multi-agent systems.
\scntext{аннотация}{Это первый учебник, специально разработанный для использования в качестве учебника для студентов и аспирантов по многоагентным системам. Предполагая лишь базовое понимание информатики, этот текст представляет собой введение во все основные вопросы теории и практики интеллектуальных агентов и многоагентных систем.}

% Another issue is how to perform the decomposition. One possibility is that the problem is decomposed by one individual agent. However, this assumes that this agent must have the appropriate expertise to do this — it must have knowledge of the task structure, that is, how the task is 'put together'. If other agents have knowledge pertaining to the task structure, then they may be able to assist in identifying a better decomposition. The decomposition itself may therefore be better treated as a cooperative activity.
\scntext{цитата}{...Другой вопрос как выполнить декомпозицию. Одна из возможностей заключается в том, что проблема разделяется на меньшие одним отдельным агентом. Однако такое решение предполагает, что этот агент должен иметь соответствующее знание, чтобы сделать это — он должен знать структуру задачи, то есть то, как задача «складывается». Если другие агенты обладают знаниями, относящимися к структуре задачи, они могут помочь определить наилучший способ декомпозиции. Таким образом, саму декомпозицию можно рассматривать как совместную деятельность...}
\scnaddlevel{1}
    \scnrelto{пояснение}{Библиотека координирующих sc-агентов}
\scnaddlevel{-1}

% https://books.google.by/books?id=kM6NDwAAQBAJ&printsec=frontcover&dq=isbn:9781420082791&hl=ru&source=gbs_book_other_versions#v=onepage&q&f=false
\scnheader{Jones.R.GarbagCollecHandbookAAMM-2016кн}
\scnrelfrom{тип}{книга}
\scnrelfromlist{ключевой знак}{информационный мусор;сборщик мусора;память}

% Published in 1996, Richard Jones’s Garbage Collection was a milestone in the area of automatic memory management. The field has grown considerably since then, sparking a need for an updated look at the latest state-of-the-art developments. The Garbage Collection Handbook: The Art of Automatic Memory Management brings together a wealth of knowledge gathered by automatic memory management researchers and developers over the past fifty years. The authors compare the most important approaches and state-of-the-art techniques in a single, accessible framework.

\scntext{аннотация}{Опубликованная в 1996 году книга Ричарда Джонса Garbage Collection стала важной вехой в области автоматического управления памятью. С тех пор эта область значительно расширилась, что вызвало потребность в обновленном взгляде на последние современные разработки. Книга объединяет обширные знания, собранные исследователями и разработчиками автоматического управления памятью за последние пятьдесят лет. Авторы сравнивают наиболее важные подходы и современные методы в единой доступной структуре.}

\scntext{цитата}{Автоматическое динамическое управление памятью разрешает множество проблем. Сборка мусора не допускает создания "оборванных"  указателей: сущность удаляется лишь тогда, когда к ней не существует указателя из другой достижимой сущности. Таким образом, гарантируется, что любой мусор будет удалён -- все недосягаемые сущности будут так или иначе удалены сборщиком мусора.}
\scnaddlevel{1}
    \scnrelto{пояснение}{Библиотека sc-агентов обнаружения и удаления информационного мусора}
\scnaddlevel{-1}

% https://e-notabene.ru/kp/article_8299.html#:~:text=Агенты%20логического%20вывода.%20К%20данному,использовать%20в%20прикладной%20интеллектуальной%20системе
\scnheader{Шункевич.Д.В.МногоагПодходКПострМашинОбрЗнОсновСемСет-2013ст}
\scnrelfrom{тип}{статья}
\scnrelfromlist{ключевой знак}{семантические сети; агенты; обработка знаний; проектирование; технология; машины; многоагентные системы; информационно-поисковая машина; операции; задачи}
\scntext{аннотация}{В данной работе рассматриваются проблемы существующих мето-дов, средств и технологий построения машин обработки знаний, рас-сматривается подход к их построению, призванный решить поставлен-ную проблему путем интеграции различных методов и способов реше-ния задач на общей формальной основе. Машина обработки знаний каждой конкретной системы во многом зависит от назначения данной системы, множества решаемых задач. Основная проблема, рассматриваемая в данной работе, заключается в отсутствии средств, позволяющих относительно неподготовленному разработчику в удовлетворительные сроки проектировать машины обработки знаний для прикладных интеллектуальных систем различного назначения. Технология проектирования машин обработки знаний предполагает использование многоагентной архитектуры. В данной работе рассматриваются два основных способа классификации агентов: по функциональному назначению и по внутренней структуре. Рассматривается ряд средств, обеспечивающих дополнительные возможности при проектировании машин обработки знаний на основе библиотек. В статье рассмотрены наиболее значимые средства, позволяющие осуществлять проектирование многоагентных систем.}

\scntext{цитата}{Агенты логического вывода. К данному классу относятся агенты, предназначенные для генерации новых знаний на основе некоторых логических утверждений. Количество и разнообразие таких агентов зависит от типологии логических утверждений, которые предполагается использовать в прикладной интеллектуальной системе.}
\scnaddlevel{1}
    \scnrelto{опрделение}{Агент логического вывода}
    \scnrelto{пояснение}{Библиотека sc-агентов логического вывода}
\scnaddlevel{-1}

% https://libeldoc.bsuir.by/bitstream/123456789/34972/1/Golenkov_Semanticheskiye.PDF
\scnheader{Голенков.В.В.СеманТехнПроектИнтелСистСеманАссоцКомп-2019ст}
\scnrelfrom{тип}{статья}
\scnrelfromlist{ключевой знак}{технология OSTIS;семантический ассоциативный компьютер;SC-код}
\scntext{аннотация}{В статье проведен анализ проблемы обеспечения совместимости компьютерных систем, рассмотрены основные принципы, лежащие в основе технологии OSTIS, одной из задач которой является решение данной проблемы. Отдельное внимание уделено принципам построения семантических ассоциативных компьютеров, являющихся аппаратной реализацией интерпретатора логико-семантических моделей компьютерных систем, разрабатываемых по технологии OSTIS.}

\scntext{цитата}{Одной из важнейших особенностей систем, построенных на основе технологии OSTIS (ostis-систем), является их платформенная независимость, которая достигается путем четкого разделения унифицированной логико-семантической модели такой системы (sc-модели компьютерной системы) и универсального интерпретатора sc-моделей компьютерных систем. Реализация универсального интерпретатора sc-моделей компьютерных систем может иметь большое число вариантов – как программно, так и аппаратно реализованных. Логическая архитектура универсального интерпретатора sc-моделей компьютерных систем обеспечивает независимость проектируемых компьютерных систем от многообразия вариантов реализации интерпретатора их моделей...}
\scnaddlevel{1}
    \scnrelto{пояснение}{Библиотека платформ интерпретации sc-моделей компьютерных систем}
\scnaddlevel{-1}

%https://libeldoc.bsuir.by/bitstream/123456789/3926/1/Golenkov_Semanticheskaya.PDF
\scnheader{Голенков.В.В.СеманТехнКомпонПроектСистУпрЗн-2015ст}
\scnrelfrom{тип}{статья}
\scnrelfromlist{ключевой знак}{системы, управляемые знаниями;компонентное проектирование;технология проектирования;представление знаний}
\scntext{аннотация}{В работе рассматривается итог пятилетнего развития Проекта OSTIS, направленного на создание Открытой семантической технологии проектирования интеллектуальных систем. В основе указанной технологии лежит представление знаний в виде унифицированных семантических сетей с теоретико-множественной интерпретацией. В работе рассматриваются классы систем, основанных на знаниях, и систем, управляемых знаниями.}

\scntext{цитата}{Компьютерные системы, управляемые знаниями, могут легко интегрировать новые полезные для них компоненты, вводимые в состав постоянно расширяемых библиотек многократно используемых компонентов. Таким образом, каждая компьютерная система, управляемая знаниями, может эволюционировать в ходе своей эксплуатации как в результате деятельности своих разработчиков, которые должны постоянно её совершенствовать, так и на основе постоянно расширяемых библиотек многократно используемых компонентов.}
\scnaddlevel{1}
    \scnrelto{пояснение}{Библиотека OSTIS}
\scnaddlevel{-1}

\scntext{цитата}{...внутренние агенты, каждый из которых реагирует на определенного вида ситуации или события в графодинамической памяти и осуществляет изменение состояния графодинамической памяти, соответствующее своему функциональному назначению...}
\scnaddlevel{1}
        \scnrelto{пояснение}{Библиотека многократно используемых внутренних агентов ostis-систем}
\scnaddlevel{-1}

\scntext{цитата}{Унифицированный способ кодирования различных видов знаний необходим для обеспечения совместимости различных видов знаний, что, в свою очередь необходимо для интеграции различных языков представления баз знаний. Без приведения интегрируемых знаний к одинаковой (общей, унифицированной) форме интеграция невозможна.}
\scnaddlevel{1}
    \scnrelto{пояснение}{Библиотека sc-агентов погружения интегрируемого знания в базу знаний}
\scnaddlevel{-1}

% https://libeldoc.bsuir.by/bitstream/123456789/34156/1/avtoreferat_Davydenko.pdf
\scnheader{Давыденко.И.Т.МоделМетодСредствРазрГибридБЗОССМИК-2018дис}
\scnrelfrom{тип}{диссертация}
\scnrelfromlist{ключевой знак}{база знаний;система, основанная на знаниях;семантическая сеть;семантическая структуризация;семантическая совместимость
баз знаний}
\scntext{аннотация}{Разработка моделей, методики и средств согласованного построения и модификации гибридных баз знаний, представленных
в виде семантических сетей, на основе семантической совместимости многократно используемых компонентов}

\scntext{цитата}{Библиотека многократно используемых компонентов баз знаний включает в себя сами компоненты, средства их спецификации и средства автоматизации их поиска на основе указанных спецификаций. Каждый многократно
используемый компонент баз знаний с формальной точки зрения представляет
собой структуру. К основным семантическим классам многократно используемых компонентов баз знаний относятся спецификации различных сущностей,
онтологии различных предметных областей, базы знаний типовых подсистем,
интегрируемых в состав разрабатываемых интеллектуальных систем.}
\scnaddlevel{1}
    \scnrelto{пояснение}{Библиотека многократно используемых компонентов баз знаний ostis-систем}
\scnaddlevel{-1}

% http://proc.ostis.net/articles/2.pdf
\scnheader{Голенков.В.В.ПроектОткрСемнТехнКомпПроектИСЧ2УМП-2014ст}
\scnrelfrom{тип}{статья}
\scnrelfromlist{ключевой знак}{интеллектуальная система;технология проектирования интеллектуальных систем;унифицированная семантическая сеть;язык семантических сетей;компонентное проектирование интеллектуальных систем;графовый язык программирования;интеллектуальная метасистема}
\scntext{аннотация}{Статья является второй в цикле статей, посвященных рассмотрению открытого проекта, направленного на создание и развитие технологии компонентного проектирования интеллектуальных
систем. В работе рассматривается унификация семантических моделей обработки знаний - моделей информационного поиска, моделей интеграции знаний, моделей решения задач, моделей
трансляции семантических сетей во внешнее представление и обратно. На основе унифицированных семантических моделей интеллектуальных систем рассмотрена модель их компонентного
проектирования, основанная на выделении многократно используемых компонентов интеллектуальных систем и на обеспечении платформенной независимости их проектирования. Рассмотрены
также средства обеспечения открытого характера технологии проектирования интеллектуальных
систем, методика их эволюционного проектирования и принципы построения метасистемы, предназначенной для комплексной поддержки проектирования интеллектуальных систем.}
\scntext{цитата}{Семейство информационно-поисковых sc-агентов, каждый из которых реагирует на соответствующий ему тип sc-вопроса (который при этом должен быть инициирован) и выполняет соответствующую поисковую процедуру в sc-памяти.}
\scnaddlevel{1}
    \scnrelto{пояснение}{Библиотека sc-агентов информационного поиска}
\scnaddlevel{-1}

\scnheader{Ивашенко.В.П.МоделАлгорИнтегрЗнОснОднорСемСет-2015дис}
\scnrelfrom{тип}{диссертация}
\scnrelfromlist{ключевой знак}{база знаний;онтология;семантическая интеграция данных;отображение и выравнивание онтологий;интеграция знаний;графовые языки;однородные семантические сети;спецификация программы;модели логического вывода;графодинамическая модель обработки информации;абстрактная машина}
\scntext{аннотация}{В диссертации исследуются онтологии и фрагменты баз знаний, представленные однородными семантическими сетями специального вида, и отношения их интеграции, возникающие на этапах создания и функционирования соответствующих интеллектуальных систем. Целью работы является разработка моделей, алгоритмов и программных средств, обеспечивающих точность и относительную полноту результатов интеграции такого рода знаний.}

\scntext{цитата}{Интеграция разработанных фрагментов баз знаний осуществляется с помощью программных средств, поддерживающих редактирование и проверку разрабатываемой базы знаний. В рамках программных средств предусмотрены программные компоненты, позволяющие преобразить исходные тексты в множество фрагментов семантической сети, осуществить проверку и интеграцию этих фрагментов и оценить качество полученных результатов. Реализованные в рамках программных средств алгоритмы интеграции и отладки баз знаний могут быть использованы для постороения и обработки распределенных баз знаний в интеллектуальных системах, поддерживающих различные механизмы решения задач.}
\scnaddlevel{1}
    \scnrelto{пояснение}{Библиотека sc-агентов погружения интегрируемого знания в базу знаний}
\scnaddlevel{-1}

% https://libeldoc.bsuir.by/bitstream/123456789/37841/1/Golenkov_Ontologicheskoye.pdf
\scnheader{Голенков.В.В.ОнтолПроектГибрСемСовмИСОснСмыслПредЗн-2019ст}
\scnrelfrom{тип}{статья}
\scnrelfromlist{ключевой знак}{интеллектуальная система;язык представления знаний;модель
интеллектуальной системы;компонентное проектирование;совместимость систем}
\scntext{аннотация}{Работа посвящена проблеме обеспечения семантической совместимости интеллектуальных систем. Показано, что обеспечение совместимости интеллектуальных систем и разработка
соответствующих стандартов является ключевым направлением развития технологий
проектирования интеллектуальных систем. Формально уточнено понятие смыслового
представления информации в памяти интеллектуальной системы, которое обеспечивает
однозначность представления информации с использованием заданного набора понятий. Показана
возможность автоматической интеграции знаний в рамках смыслового представления знаний,
которая сводится к склеиванию синонимичных знаков. Показана возможность автоматической
интеграции различных моделей обработки знаний, если эти модели представляют собой
коллективы агентов, ориентированных на обработку знаний, представленных в памяти
интеллектуальных систем в смысловой форме, и взаимодействующих между собой через
указанную память. Предложена Технология OSTIS, ориентированная на разработку
семантических компьютерных систем. Предложена концепция Экосистемы OSTIS,
представляющей собой коллектив взаимодействующих интеллектуальных систем, построенных по
Технологии OSTIS и поддерживающих эволюцию и совместимость интеллектуальных систем в
ходе их эксплуатации в рамках данной экосистемы. Рассмотрены примеры использования
Технологии OSTIS при разработке прикладных интеллектуальных систем.}

\scntext{цитата}{...процесс выравнивания понятий, целью которого является сведение всех
понятий, используемых в интегрируемом sc-тексте, к согласованным понятиям БЗ...}
\scnaddlevel{1}
    \scnrelto{пояснение}{Библиотека sc-агентов выравнивания онтологии интегрируемого знания с основной онтологией текущего состояния базы знаний}
\scnaddlevel{-1}

% https://ssrlab.by/wp-content/uploads/2014/OSTIS-2014.pdf
\scnheader{Шункевич.Д.В.МашинОбрЗнИнтелМетаСистПоддержПроектИнтелСист-2014ст}
\scnrelfrom{тип}{статья}
\scnrelfromlist{ключевой знак}{машина обработки знаний;интеллектуальная система;многоагентная система;технология проектирования}
\scntext{аннотация}{Данная статья посвящена принципам организации и функционирования машин обработки знаний
интеллектуальных систем в целом, лежащим в основе технологии проектирования машин обработки знаний в
рамках технологии OSTIS. В качестве примера применения такой технологии рассматривается машина
обработки знаний интеллектуальной метасистемы IMS.OSTIS направленной на поддержку проектирования
интеллектуальных систем}

\scntext{цитата}{В общем случае один sc-агент может явно
передать управление другому sc-агенту, если этот
sc-агент априори известен. Для этого каждый scагент в sc-памяти имеет обозначающий его sc-узел,
с которым можно связать конкретную ситуацию в
текущем состоянии базы знаний, которую
инициируемый sc-агент должен обработать.
}
\scnaddlevel{1}
    \scnrelto{пояснение}{Библиотека sc-агентов планирования решения явно сформулированных задач}
\scnaddlevel{-1}

\end{small}
\end{SCn}